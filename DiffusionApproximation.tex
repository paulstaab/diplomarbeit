\section{The Diffusion Approximation}

In this chapter, we are using diffusion theory to approximate Muller's ratchet (with compensatory
mutations) by a time continuous stochastic process. Similar as it has been done for the original
ratchet (e.g. in \cite{stephan_advance_1993}, \citet{etheridge_how_2008}), we translate the
dynamics in the definition of Muller's ratchet with compensatory mutations into a stochastic differential
equation (SDE) on $\Simplex$ defined by
\begin{align} \label{da:eq:difdef}\tag{$\ast$}
d \mrDifKT{k}{t} = 
	b_k(\mrDifT{t}) dt 
  + \sum_{l \not= k} \sqrt{\frac{1}{N} \mrDifKT{k}{t} \mrDifKT{l}{t} } \, dW_{kl}
\end{align}
for $k \in \N $, where $X_{-1} := 0$, 
\begin{align} \label{da:eq:difdefdet}
 b_k(\Vector{x}) := 	
 \alpha \sum_l (l-k) x_k x_l 
  + \lambda \left( x_{k-1} - x_k \right)
  + \gamma \left( (k+1) x_{k+1} - k x_k \right) 
\end{align}
and $W_{kl}$ are independent, $1$-dimensional Brownian motions for $k>l$ and
$W_{kl} := -W_{lk}$ for $k<l$.
For $\alpha = \lambda = \gamma = 0$, this is the classical diffusion approximation of
the Wright-Fisher dynamics. Again, we add selection, mutation and compensatory mutations to this
model via the three corresponding terms in \eqref{da:eq:difdefdet}. As $(1-\alpha)^k \sim 1 - \alpha
k$ for small $\alpha$, we have the occurrence of $-\alpha k$ in the selection term. The mutation
term reflects the flow of individuals of type $k$ to type $k+1$ and from $k-1$ to $k$, both with
constant rate $\lambda$. The compensatory mutation term is similar, but the rates depend on the
current number of mutations. We briefly recall the necessary notations for
stochastic differential equations.

\begin{Definition}[SDE]
We say that a tuple $(\Omega,\mathscr{F},\mathbf{P},(W_{kl})_{k<l},\mrDif)$,
consisting of a filtered probability space $(\Omega,\mathscr{F},\mathbf{P})$, an array
$(W_{kl})_{k<l}$ of $\mathscr{F}$-Brownian motions satisfying the conditions
mentioned above and an adapted process $\mrDif$, is a \emph{(weak) solution} of
the SDE~\eqref{da:eq:difdef} with initial distribution $\mu$, if
$X(0)_*\mathbf{P} = \mu$ and it satisfies the SDE or, more explicitly,
\[ 
\mrDifKT{k}{t} = \mrDifKT{k}{0} + \int_0^t b_k(\mrDifT{s}) \, ds 
  + \sum_{l \not= k} \int_0^t \sqrt{\frac{1}{N} \mrDifKT{k}{s} \mrDifKT{l}{s} }
  \, dW_{kl}(s)
\]
for all $k \in \N$. We say that \emph{(weak) existence} holds for an initial
distribution if there is a weak solution with this initial distribution.
Analogous, \emph{uniqueness} holds for an initial distribution if all weak
solutions for it are equal in distribution. We say the SDE is \emph{well-posed}
if existence and uniqueness holds for every initial distribution. For Muller's
ratchet, we always assume that $\mathscr{F}$ is the natural filtration of $\mrDif$.
\end{Definition}

\noindent
It is a well-known result by Stroock and Varadhan that stochastic differential equations can
equivalently be formulated as martingale problems under sufficient conditions (e.g.
\cite[Theorem 18.7]{kallenberg_foundations_1997}). The diffusion approximation of Muller's ratchet
was defined as a SDE, because this is quite typical for Wright-Fisher models.
However, a ``typical'' Wright-Fisher model has only a finite number of different
types, whereas the type space of Muller's ratchet is $\N$. Therefore it is realised as
a Fleming-Viot process, which are typically defined as martingale problems (e.g.
\cite{ethier_flemingviot_1993}). Hence, we want to formulate Muller's ratchet as
a martingale problem in order to apply results from the theory of Fleming-Viot
processes. This requires further investigation, as the conditions of the
Stroock-Varadhan-Theorem are not given here, mainly because $\Simplex$ is an
infinite dimensional space.


\subsection{Characterisation As A Martingale Problem}

As we have done for stochastic different equations, we start by recalling the basic notations in the
theory of martingale problems.

\begin{Definition}[Martingale problem]
Let $(E,r)$ be a complete and separable metric space, $\mathbf{P}_0 \in \mathcal{P}(E)$,
$\mathcal{F} \subseteq \mathcal{M}(E)$ and $G$ a linear operator on
$\mathcal{M}(E)$. A distribution $\mathbf{P}$ of an $E$-valued stochastic process $\mathcal{X} =
(\mathcal{X}_t)_{t \geq 0}$ is called a solution of the $(E,
\mathbf{P}_0,G,\mathcal{F})$-martingale problem if $\mathcal{X}_0$ has
distribution $\mathbf{P}_0$, almost all paths of $\mathcal{X}$ are cadlag
and $N^f = (N^f_t)_{t \geq 0}$ defined by
\begin{align} \label{da:eq:N^f}
N^f_t = 
f(\mathcal{X}_t) - f(\mathcal{X}_0) - \int_0^t Gf(\mathcal{X}_s) ds
\end{align}
is a $\mathbf{P}$-martingale with respect to the canonical filtration for all $f \in \mathcal{F}$.
We say that a process $\mathcal{X}$ solves the $(E,
\mathbf{P},G,\mathcal{F})$-martingale problem if its distribution does so and refer to $G$ as its (infinitesimal) generator. Moreover, we call a
martingale problem well-posed if there is a unique solution $\mathbf{P}$. 
\end{Definition}

\noindent
Hence, we need a domain $\mathcal{F}$ and a generator $G$ to define a martingale problem
on $\Simplex$. We define the ones that correspond to SDE~\eqref{da:eq:difdef} in the next
definition. In fact, we state two different domains $\mathcal{F}$ and $\overline{\mathcal{F}}$ for
technical reasons. We will show in Theorem~\ref{da:t:SDE=MP} that the
corresponding martingale problems are equivalent.

\begin{Definition}[Domains and generator for Muller's Ratchet]\mbox{}\label{da:d:D&G-MP}
Recall Definition~\ref{pre:d:F}. In addition to $\overline{\mathcal F}$, we
define the algebra
\begin{align*}
\mathcal{F} &:=  \mathcal{A} \left( \{f_{\varphi_1,...,\varphi_n}: n\in\mathbb N, \varphi_i \in
        	\FB{\N} \text{ with } |\text{supp}(\varphi_i)| < \infty \text{ for } i = 1,\ldots,n\}
        	\right) \\
\end{align*}

\noindent
and the generator $\GA$ on $\overline{\mathcal F}$ with domain $\overline{\mathcal F}$
by

{\allowdisplaybreaks
\begin{equation*} \label{da:eq:GA}
\begin{aligned}
\GAFX{f}{\Vector{x}} & =
		\GAselFX{f}{\Vector{x}} + \GAmutFX{f}{\Vector{x}} 
	    + \GAcompFX{f}{\Vector{x}} + \GNresFX{f}{\Vector{x}} \\
\GAselFX{f}{\Vector{x}} & =
        \alpha \sum_{k=0}^\infty \sum_{\ell=0}^\infty(\ell-k)x_\ell x_k
        \frac{\partial}{\partial x_k} f(\Vector{x}),\\ 
\GAmutFX{f}{\Vector{x}} & = 
		\lambda \sum_{k=0}^\infty (x_{k-1} - x_k) \frac{\partial}{\partial x_k} f(\Vector{x}),\\
\GAcompFX{f}{\Vector{x}} & = 
		\gamma \sum_{k=0}^\infty ((k+1)x_{k+1} - kx_k) \frac{\partial}{\partial x_k}
		f(\Vector{x}) \qquad \text{and}\\ 
\GNresFX{f}{\Vector{x}} &= 
		\tfrac 1{2N} \sum_{k,\ell = 0}^\infty x_k ( \delta_{kl} - x_\ell) \frac{\partial^2}{\partial x_k
		\partial x_\ell} f(\Vector{x})
      \end{aligned}
\end{equation*}}

\noindent
with $\alpha, \lambda, \gamma\in [0,\infty)$, $N\in(0,\infty)$.
\end{Definition}

% \begin{Remark} XXX Wrong! XXX
% Please note that $\overline{\mathcal F}$ contains functions defined on $\R^\N$ and the partial
% derivatives inside the generator are the normal ones on this space. However, we will use $\Simplex$
% and $\SimplexEM$ as state spaces for our martingale problems. In that case, we restrict the
% functions in $\overline{\mathcal F}$ to $\Simplex$ or $\SimplexEM$, eventually \emph{after} partial
% derivation.
% \end{Remark}

\noindent
We will now prove a lemma that is the technical key for the diffusion approximation.
Basically, it says that if Muller's ratchet starts with a distribution in
$\SimplexEM$, then it will always take values in this space.

\begin{Lemma}[Bounds on exponential moments]\label{da:l:expBounds}
Let $\Vector{x} \in \SimplexEM$ and $\mrDif =
(\mrDifT{t})_{t\geq 0}$ be a solution of the $(\Simplex,
\delta_{\Vector{x}},\GA, \mathcal{F})$-martingale
problem and recall that $h_\xi(\Vector{x}):=\sum_{k \in \N} x_k e^{\xi k}$. Then, for all $T>0$,
there is $C>0$, depending on $T$ and all model parameters with
\begin{align}\label{da:eq:expBounds1}
  \mathbf E\big[ & \sup_{0\leq t\leq T} h_\xi(\mrDifT{t})^2\big] \leq C \cdot
  h_{2\xi+2}(\Vector{x}).
\end{align}
In particular, we have $\sup_{0\leq t\leq T} h_\xi(\mrDifT{t}) < \infty$ almost
surely.
\end{Lemma}

\begin{proof}
We define 
$$ h_{\xi,m}(\underline x) := \sum_{k=0}^m x_k e^{\xi k}
+ e^{\xi m} \left( 1 - \sum_{k=0}^m x_k \right)
= e^{\xi m} + \sum_{k=0}^m x_l (e^{\xi k} - e^{\xi m})
\in \mathcal{F}
$$ 
for $m \in \N$. Note that for $\Vector{x} \in \Simplex$,
$$ h_{\xi,m}(\underline x) = \sum_{k=0}^\infty x_k e^{\xi(k \wedge m)} $$
We compute
\begin{equation} \label{da:eq:imp1}
\begin{aligned}
\GAmutFX{h_{\xi,m}}{\Vector{x}} 
& = \lambda \sum_{k=0}^m (x_{k-1} - x_k) (e^{\xi k} - e^{\xi m}) \\ 
&  = \lambda \sum_{k=0}^{m-1} x_k \Big( e^{\xi(k+1)} - e^{\xi k}\Big) \geq 0,\\
\GAcompFX{h_{\xi,m}}{\Vector{x}} 
& = \lambda \sum_{k=0}^m ((k+1)x_{k+1} - k x_k) (e^{\xi k} - e^{\xi m}) \\ 
&  = \lambda \sum_{k=0}^m k x_k \Big( e^{\xi(k-1)} - e^{\xi k}\Big) \leq 0,\\
\GAselFX{h_{\xi,m}}{\Vector{x}} 
& = \alpha \sum_{k=0}^m \sum_{\ell=0}^\infty (\ell-k) x_\ell x_k (e^{\xi k} - e^{\xi m}) \\
& = \alpha \sum_{k=0}^\infty \sum_{\ell=0}^\infty (\ell-k) x_\ell x_k (e^{\xi (k \wedge m)} - e^{\xi
m}) \\ 
& = \alpha \sum_{k=0}^\infty \sum_{\ell=0}^\infty (\ell-k) x_\ell x_k e^{\xi (k \wedge m)} \leq 0,
\end{aligned}
\end{equation}
where the latter equality only holds for $x\in\Simplex$. For the final
inequality, assume that $Z$ is an $\N$-valued random variable with
distribution $\Vector{x}$. 
Then we have 
$$ \GAselFX{h_{\xi,m}}{\Vector{x}} 
= - \alpha \, \mathbf{COV}[Z, e^{\xi (Z\wedge m)}] \leq 0,$$ 
since for non-decreasing functions $f,g$ and
i.i.d. random variables $Z$ and $Z'$
$$ \E{\big(f(Z) - f(Z')\big)\big(g(Z) - g(Z')\big)} \geq 0 $$
as both factors have the same sign. It follows that
$$ 2 \, \mathbf{COV}[f(Z),g(Z)] = 2 \, \E{f(Z)g(Z)} - 2 \, \E{f(Z)} \E{g(Z)}
\geq 0 .$$

\noindent
Second, we prove that a similar bound as
\eqref{da:eq:expBounds1} holds at fixed time points. We write
\begin{align*}
\frac{d}{dt} \mathbf E[h_{\xi,m}(\mrDifT{t})] 
& = \mathbf E\big[ \GAFX{h_{\xi,m}}{\mrDifT{t}} \big] 
  \leq \mathbf E\big[ \GAmutFX{h_{\xi,m}}{\mrDifT{t}} \big] \\ 
& \leq \lambda \mathbf E\Big[ e^\xi \sum_{k=0}^\infty \mrDifKT{k}{t} 
		e^{\xi(k\wedge m)} - \sum_{k=0}^\infty X_k(t) e^{\xi(k\wedge m)}\Big] \\ 
& = \lambda (e^\xi - 1) \mathbf E[h_{\xi,m}(\mrDifT{t})].
\end{align*}
So, by Gronwall's inequality,
\begin{align*}%\label{eq:gron1}
\mathbf E[h_{\xi,m}(\mrDifT{t})] 
  \leq h_{\xi,m}(\Vector{x}) \cdot \exp\big( \lambda t(e^\xi - 1)\big)
\end{align*}
which also implies 
\begin{align}\label{da:eq:gron1}
\mathbf E[h_{\xi}(\mrDifT{t})] \leq h_{\xi}(\Vector{x}) \cdot \exp\big( \lambda t(e^\xi - 1)\big)
\end{align}
by monotone convergence. 

\noindent
Third, \eqref{da:eq:imp1} implies that
$$ \left( h_{\xi,m}(\mrDifT{t}) - 
	\int_0^t \GAselFX{h_{\xi,m}}{\mrDifT{s}} + \GAcompFX{h_{\xi,m}}{\mrDifT{s}} ds \right)_{t\geq 0} $$
is a submartingale. Now, by Doob's submartingale inequality and the fact that 
$(h_\xi(\underline x))^2 \leq h_{2\xi}(\underline x)$ for all $\underline x\in\mathbb S$ and
$\xi\geq 0$,
 
\begin{align*}
\mathbf E[&\sup_{0\leq t\leq T} (h_{\xi,m} (\mrDifT{t}))^2]\\
&  \leq \mathbf E\Big[\sup_{0\leq t\leq T} \Big(h_{\xi,m} (\mrDifT{t}) 
  	- \int_0^t \GAselFX{h_{\xi,m}}{\mrDifT{s}} + \GAmutFX{h_{\xi,m}}{\mrDifT{s}} ds\Big)^2\Big] \\
& \leq 4 \cdot \mathbf E\Big[\Big(h_{\xi,m} (\mrDifT{T}) 
	- \int_0^T \GAFX{h_{\xi,m}}{\mrDifT{s}} + h_{\xi,m}(\mrDifT{s}) ds\Big)^2\Big] \\ 
& \leq 8 \cdot \Big(\mathbf E\big[ (h_{\xi,m}(\mrDifT{T}))^2\big] \\ 
& \qquad \qquad + \int_0^T \int_0^T \mathbf E[(\GAselFX{h_{\xi,m}}{\mrDifT{s}}+
											   \GAcompFX{h_{\xi,m}}{\mrDifT{s}}) \\ 
& \qquad \qquad \qquad \qquad \qquad \qquad \cdot (\GAselFX{h_{\xi,m}}{\mrDifT{r}}+
											   \GAcompFX{h_{\xi,m}}{\mrDifT{r}})dr ds\Big) \\ 
& \leq 8 \cdot \Big(\mathbf E\big[ h_{2\xi}(\mrDifT{T})\big] + (\alpha + \gamma)^2 
	\int_0 ^T \int_0^T \mathbf E[h_{\xi+1}(\mrDifT{s}) h_{\xi+1}(\mrDifT{r})] dr ds \Big)\\ 
& \leq 8 \cdot \mathbf E\big[ h_{2\xi}(\mrDifT{T})\big] + 16 \cdot (\alpha + \gamma)^2 T
	\int_0^T \mathbf E\big[ h_{2\xi+2}(\mrDifT{s})\big]ds \\ 
& \leq C\cdot h_{2\xi+2}(\underline x)
\end{align*}
for $C = (8+16\alpha^2 T^2) \exp\big(\lambda T(e^{2\xi+2}-1)\big)$ by \eqref{da:eq:gron1}.
\end{proof}

\noindent
We need another small result to prove Theorem~\ref{da:t:SDE=MP}.

\begin{Lemma} \label{da:l:f'}
Let $f_{\varphi_1,\ldots,\varphi_n}$ be as in Definition~\ref{pre:d:F}. For $k,l \in \N$, its 
first and second order partial derivatives are 
\begin{align*}
\frac{\partial}{\partial x_k} f_{\varphi_1,\ldots,\varphi_n}(\Vector{x}) 
&= 	\sum_{i=1}^n \varphi_i(k)
f_{\varphi_1,\ldots,\widehat{\varphi_i},\ldots,\varphi_n}(\Vector{x})
\end{align*}
and
\begin{align*}
\frac{\partial^2}{\partial x_k \partial x_l} f_{\varphi_1,\ldots,\varphi_n} (\Vector{x})
= \sum_{\substack{i,j=1,\\i \not = j}}^n \varphi_i(k) \varphi_j(l)
f_{\varphi_1,\ldots,\widehat{\varphi_i},\ldots,\widehat{\varphi_j},\ldots,\varphi_n}(\Vector{x}) 
\end{align*}
where $\widehat{\varphi_i} := 1$ for all $i$.
\end{Lemma}

\begin{proof}
We use an induction over $n$. While
$n=1$ is trivial, we have
\begin{align*}
\frac{\partial}{\partial x_k} f_{\varphi_1,\ldots,\varphi_n}(\Vector{x})
&=	\frac{\partial}{\partial x_k} \left( f_{\varphi_1,\ldots,\varphi_{n-1}}(\Vector{x}) 
	\sum_{i=0}^\infty x_i \varphi_n(i) \right) \\
&=	\sum_{i=1}^{n-1} \varphi_i(k)
f_{\varphi_1,\ldots,\widehat{\varphi_i},\ldots,\varphi_{n-1}}(\Vector{x})
	\sum_{i=0}^\infty x_i \varphi_n(i) \\
&\quad + f_{\varphi_1,\ldots,\varphi_{n-1}}(\Vector{x}) \varphi_n(k)
	 \\
&= 	\sum_{i=1}^n \varphi_i(k) f_{\varphi_1,\ldots,\widehat{\varphi_i},\ldots,\varphi_n}(\Vector{x}).
\end{align*}
For the second order derivatives, we calculate
\begin{align*}
\frac{\partial^2}{\partial x_k \partial x_l} f(\Vector{x})
&=	\frac{\partial}{\partial x_l} \sum_{i=1}^n \varphi_i(k) 
	f_{\varphi_1,\ldots,\widehat{\varphi_i},\ldots,\varphi_n}(\Vector{x}) \\
&= \sum_{\substack{i,j=1,\\i \not = j}}^n \varphi_i(k) \varphi_j(l)
f_{\varphi_1,\ldots,\widehat{\varphi_i},\ldots,\widehat{\varphi_j},\ldots,\varphi_n}(\Vector{x}) 
\qedhere
\end{align*}
\end{proof}

\noindent
We define the following notation to abbreviate calculations with stochastic integrals.  

\begin{Notation}
For suitable stochastic processes $H$ and $X$, we denote the \Ito integral with
\[ \int H_s d X_s = H \cdot X. \] 
To evaluate the integral on time $t$, we write
\[ \int_0^t H_s d X_s = H \cdot X_t. \]
Analogously, we denote the Lebesgue-measure with $\nu$ and the integral of $f: \R \to \R$ from $0$
to $t$ as
\[ \int_0^t f(s) ds = f \cdot \nu(t) \]
\end{Notation}

\noindent
After these necessary preparations, we are now ready to prove the maybe most technical theorem of
this manuscript.

\begin{samepage}
\begin{Theorem} \label{da:t:SDE=MP}
Let $\mrDif$ be a process with values in $\Simplex$ and $\mrDifT{0} = \Vector{x} \in
\SimplexEM$. Then, the following are equivalent:
\begin{enumerate}
\item The process $\mrDif$ solves the $(\Simplex,\delta_{\Vector{x}},\GA,\mathcal{F})$-martingale problem.
\item The process $\mrDif$ solves the
$(\SimplexEM,\delta_{\Vector{x}},\GA,\overline{\mathcal{F}})$-martingale problem.
\item The process $\mrDif$ is a weak solution of the SDE \eqref{da:eq:difdef} with initial
distribution $\delta_{\Vector{x}}$.
\end{enumerate}
\end{Theorem}
\end{samepage}

\begin{proof}
2. $\Rightarrow$ 1.: Obviously $\mathcal{F} \subseteq \overline{\mathcal{F}}$. The assertion follows directly.

\noindent
1. $\Rightarrow$ 2.: Let $\mrDif$ solve the
$(\Simplex,\delta_{\Vector{x}},\GA,\mathcal{F})$-martingale problem. 
By Lemma~\ref{da:l:expBounds}, $\mrDif$ takes values in $\SimplexEM$ a.s. and hence is a solution of
the $(\SimplexEM,\delta_{\Vector{x}},\GA,\overline{\mathcal{F}})$-martingale
problem if $N^f$ (as in \eqref{da:eq:N^f} with $G = \GA$) is a martingale for
every $f \in \overline{\mathcal{F}}$. However, it suffices to show the
martingale property for the generator of $\overline{\mathcal{F}}$, as it is
closed under multiplication, $N^f(t)$ is linear in $f$ and linear combinations
of martingales are martingales.

\noindent
So for a fixed $f$, there exist $\varphi_1,\ldots,\varphi_n \in
\FBEM{\Simplex}$ such that $f = f_{\varphi_1,\ldots,\varphi_n}$. We
approximate $f$ with functions $f^j := f_{\varphi^j_1,\ldots,\varphi^j_n} \in \mathcal{F}$ where $\varphi^j_i(l) :=
\varphi_i(l)\1_{\{l \leq j\}}$ with $j,l \in \N$. Hence, $f^j \xrightarrow{j\to\infty}
f$ pointwise and similar to Lemma~\ref{da:l:f'}, 
$$ \frac{\partial}{\partial x_k}f^j \xrightarrow{j\to\infty} \frac{\partial}{\partial x_k}f
\quad \text{and} \quad
\frac{\partial^2}{\partial x_k \partial x_l}f^j \xrightarrow{j\to\infty} 
\frac{\partial^2}{\partial x_k \partial x_l}f, $$
both again pointwise. 

\noindent
We now want to prove that also $\GAselFX{f^j}{\Vector{x}} \Conv \GAselFX{f}{\Vector{x}}$. Therefore, 
observe that
\begin{align*}
\sup_j | \SProd{\varphi_i^j}{\Vector{x}} |
 \leq \sum_{l=0}^\infty \sup_j | \varphi_i^j(l) | x_l 
 \leq \sum_{l=0}^\infty | \varphi_i(l) | x_l 
 \leq C \sum_{l=0}^\infty e^{\xi l} x_l 
  < \infty
\end{align*}
for some $C,\xi >0$ and all $1 \leq i \leq n$, as $\varphi_i$ is exponentially bounded and
$\Vector{x}$ has all exponential moments. Hence,
\begin{align} \label{da:eq:expboundf'}
\sup_j | \frac{\partial}{\partial x_k} f^j (\Vector{x})| 
& \leq \sup_j | \sum_{i=1}^n \varphi_i^j(k)
	f_{\varphi^j_1,\ldots,\widehat{\varphi^j_i},\ldots,\varphi^j_n}(\Vector{x}) | \\
& \leq \sum_{i=1}^n | \varphi_i(k) | C_i 
 \leq \sum_{i=1}^n \widetilde{C}_i e^{\xi_i k} 
 \leq C e^{\xi k}
\end{align}
again for suitable $C,C_i,\widetilde{C}_i,\xi_i,\xi > 0$. Now
\begin{align*}
\sum_{k=0}^\infty \sup_j | \alpha \sum_{l=0}^\infty (l-k) x_l x_k \frac{\partial}{\partial x_k} f^j
	(\Vector{x}) | 
& \leq \sum_{k=0}^\infty x_k \, C
e^{\xi k} \left| \left( \sum_{l=0}^\infty l x_l \right) - k \, \right| \\
& \leq \sum_{k=0}^\infty x_k \, \widetilde{C} e^{\widetilde{\xi} k} < \infty,
\end{align*} 
and the convergence of $\GAselFX{f^j}{\Vector{x}}$ follows by dominated convergence.
Analogously, 
\begin{align*}
\GAmutFX{f^j}{\Vector{x}} \xrightarrow{j \to \infty} \GAmutFX{f}{\Vector{x}}
\quad \text{and} \quad
\GAcompFX{f^j}{\Vector{x}} \xrightarrow{j \to \infty} \GAcompFX{f}{\Vector{x}}.
\end{align*}
Finally observe that analogous to \eqref{da:eq:expboundf'}, also
\begin{align*}
\sup_j | \frac{\partial^2}{\partial x_k \partial x_l} f_{\varphi_1^j,\ldots,\varphi_n^j}
(\Vector{x}) | \leq C e^{\xi (k+l)}
\end{align*}
and therefore
\begin{equation}
\label{da:wq:Gresbounded}
\begin{aligned}
\sum_{k,l=0}^\infty \sup_j | x_k (\delta_{kl} - x_l) \frac{\partial^2}{\partial x_k \partial x_l}
	f^j |
& \leq C \sum_{k=0}^\infty x_k e^{\xi k} \sum_{l=0}^\infty | (\delta_{kl} - x_l) e^{\xi l} | \\
& \leq \widetilde{C} \sum_{k=0}^\infty x_k e^{2 \xi k} < \infty.
\end{aligned}
\end{equation}

\noindent
This gives $\GAFX{f^j}{\Vector{x}} \to \GAFX{f}{\Vector{x}}$. Hence, $N^f$ is a martingale:
\begin{align*}
\CE{N^f \! (t)}{(X_r)_{r \leq s}} 
&= \CE{\lim_{j \to \infty} N^{f^j}\!(t)}{(X_r)_{r \leq s}} \\
&= \lim_{j \to \infty} \CE{ N^{f^j}\!(t)}{(X_r)_{r \leq s}} \\
&= \lim_{j \to \infty} N^{f^j}\!(s) \\
&= N^{f}\!(s)
\end{align*}
where we again use dominated convergence.

\noindent
3. $\Rightarrow$ 1.:
Assume that $\Vector{X}$ is a solution of the SDE \eqref{da:eq:difdef}. We
calculate the covariation process $([X_i,X_j]_t)_{t \geq 0}$ of
$X_i$ and $X_j$ as
\begin{align*}
\left[ X_i, X_j \right] 
&= \left[ \sum_{k \not = i} \sqrt{N^{-1} X_i X_k } \cdot W_{ki}, \sum_{l \not = j} \sqrt{N^{-1} X_j
X_l } \cdot W_{lj} \right]\\
&= \sum_{\substack{k \not = i\\l \not = j}} N^{-1} \sqrt{ X_i X_k X_j X_l } \cdot \left[ W_{ki},
W_{lj} \right]\\ &=
\begin{cases}
N^{-1} (-1) X_i X_j \cdot \nu & \text{for } i \not = j \\
N^{-1} X_i (1-X_i) \cdot \nu & \text{for } i = j \\
\end{cases} \\
&= N^{-1} X_i(\delta_{ij} -X_j) \cdot \nu \\
\end{align*}
Now as every $f \in \mathcal{F}$ only depends on finitely many
coordinates, we can use It\^{o}'s formula, which yields 
\begin{align*}
df(\mrDifT{t}) 
&= \sum_{i=0}^\infty f_i(\mrDifT{t})d\mrDifKT{i}{t}
 + \sum_{i,j=0}^\infty \frac{1}{2} f_{ij}(\mrDifT{t}) d[\mrDifK{i},\mrDifK{j}]_t \\
&= \sum_{i=0}^\infty f_i(\mrDifT{t}) 
		\left(b_i(\mrDifT{t}) dt + \sum_{j \not= i} a_{ij}(\mrDifT{t}) dW_{ij}\right)\\
& \qquad + \sum_{i,j=0}^\infty \frac{1}{2} f_{ij}(\mrDifT{t}) N^{-1} X_i (1-X_i) dt \\
&= \sum_{i=0}^\infty f_i(\mrDifT{t}) \sum_{j \not= i} a_{ij}(\mrDifT{t}) dW_{ij} + G^\alpha
f(\mrDifT{t})dt.
\end{align*}
with $a_{ij}(\Vector{x}) := \sqrt{\frac{1}{N} x_i x_j}$, $f_i :=
\frac{\partial}{\partial x_i} f$ and $f_{ij} := \frac{\partial^2}{\partial x_i
\partial x_j} f$. Hence,
\begin{equation}
\begin{aligned} \label{da:eq:dM^f}
dN^f_t &= df(\mrDifT{t}) - G^\alpha f(\mrDifT{t}) \\
	   &= \sum_{i=0}^\infty f_i(\mrDifT{t}) \sum_{j \not= i} a_{ij}(\mrDifT{t}) \,
	   dW_{ij}.
\end{aligned}
\end{equation}
Clearly every $a_{ij}(\mrDifT{t}) \, dW_{ij}$ is a martingale because
$a_{ij}(\mrDifT{t})$ is bounded almost surely. The same is true for $\sum_{j \not= i} a_{ij}(\mrDifT{t}) dW_{ij}$, as we can
again use dominated convergence because
\begin{align*}
&\E{\left(\sum_{j\not=i} \int_0^t \sqrt{\frac{1}{N} \mrDifKT{i}{t}
\mrDifKT{j}{t}} \, dW_{ij}\right)^2\,} \\
\leq \, &\E{\sum_{\substack{j\not=i,\\k\not=i}} 
	\left|\int_0^t \sqrt{\frac{1}{N} \mrDifKT{i}{t} \mrDifKT{j}{t}} dW_{ij}\right|
	\left|\int_0^t \sqrt{\frac{1}{N} \mrDifKT{i}{t} \mrDifKT{k}{t}} dW_{ik}\right|
	}\\
\leq \, &\sum_{\substack{j\not=i,\\k\not=i}} 
	\E{\left(\int_0^t \sqrt{\frac{1}{N} \mrDifKT{i}{t} \mrDifKT{j}{t}} dW_{ij}\right)^2}
	\E{\left(\int_0^t \sqrt{\frac{1}{N} \mrDifKT{i}{t} \mrDifKT{k}{t}} dW_{ik}\right)^2} \\
= \, &\left( \sum_{j\not=i} \E{\int_0^t \frac{1}{N} \mrDifKT{i}{s} \mrDifKT{j}{s} ds} \right)^2 \\
= \, &\left( \frac{1}{N} \Ex{\int_0^t \mrDifKT{i}{s} \left(1-\mrDifKT{i}{s}\right) ds} \right)^2 
< \infty
\end{align*}
using the Cauchy-Schwarz inequality in the 2nd step and It\^{o}'s isometry in
the 3rd one. This finishes the proof of 2. because the first sum in \eqref{da:eq:dM^f} is
effectively finite.

\noindent
2. $\Rightarrow$ 3.:
Let $\mr$ solve the $(\SimplexEM,\delta_{\Vector{x}},\GA,\overline{\mathcal{F}})$-martingale problem.  
We have to construct Brownian motions $(W_{kl})_{k \leq l}$ on $\Simplex$ such that
\eqref{da:eq:difdef} holds. To do so, we use Doob's 
integral representation theorem (e.g. \citet[Theorem
16.12]{kallenberg_foundations_1997}):

\begin{Theorem*}
Let $M$ be a continuous local $\mathscr{F}$-martingale in $\R^d$ with $M_0 = 0$
such that the covariation of two coordinate processes $M_i$ and $M_j$ is 
$$ [M_i,M_j]_t = \int_0^t \sum_{k=1}^d \sum_{l>k}^d V^i_{k,l}(s) V^j_{k,l}(s)
\, ds \quad \text{a.s.} $$ for some $\mathscr{F}$-progressive processes
$V_{k,l}^i$, $1 \leq i \leq d, 1 \leq k < l \leq d$. Then there exist some
independent standard Brownian motions $(W_{kl})_{1 \leq k < l \leq d}$ with
respect to a standard extension of $\mathscr{F}$ such that 
$M_i = \sum_{k=1}^d \sum_{l>k}^d V_{k,l}^i \cdot W_{k,l}$.
\end{Theorem*}

\noindent First notice that we can reformulate the SDE \eqref{da:eq:difdef} to
\begin{align} \label{da:eq:difNewDef} 
d \mrDifKT{i}{t} = 
	b_i(\mrDifT{t}) dt 
  + \sum_{k=0}^\infty \sum_{l > k}  (\delta_{ik} - \delta_{il}) \sqrt{\frac{1}{N} \mrDifKT{k}{t}
  \mrDifKT{l}{t} } \, dW_{kl}.
\end{align}
Analogously to the proof of Theorem 18.7 in \cite{kallenberg_foundations_1997}
we can calculate that for
\begin{align} \label{da:eq:defMi}
M_i(t) = \mrDifKT{i}{t} + \mrDifKT{i}{0} - \int_0^t b_i(\mrDifKT{i}{s}) \, ds,
\end{align}
with $b_i$ as in \eqref{da:eq:difdef}, the quadratic covariation of $M_i$ and $M_j$ is
\begin{align*}
[M_i,M_j] = \frac{1}{N} \mrDifKT{i}{s} (\delta_{ij} - \mrDifKT{j}{s}) \cdot \nu.
\end{align*}
As Doob's theorem requires $\R^d$ as state space, we summarize all coordinates
$(\mrDifK{i})$ with $i \geq R$ for fixed $R \in \N$ into a single process $\widetilde{X}_R :=
1-\sum_{i=0}^{R-1} \mrDifK{i}$ and lift the Brownian motions to $\Simplex$ with a projective limit
argument later. Furthermore, we define 
$$ \widetilde{M}_R(t) 
:=  \widetilde{X}_R(t) - \widetilde{X}_R(0) - \int_0^t b_i\left(
\widetilde{X}_R(s) \right) ds = 1-\sum_{i=0}^{R-1} M_i(t) $$
which clearly is a martingale as well. With the first line according to \eqref{da:eq:difNewDef}, we
define $$ V_{k,l}^i := \begin{cases}
(\delta_{ik} - \delta_{il}) \sqrt{\frac{1}{N} \mrDifK{k} \mrDifK{l}} 
	& \text{ for } 1 \leq i < R, 0 \leq k < l < R, \\
(\delta_{ik} - \delta_{iR}) \sqrt{\frac{1}{N} \mrDifK{k} \sum_{l \geq R} \mrDifK{l}} 
	& \text{ for } 0 \leq i < R, 0 \leq k < R, l = R, \\
0   & \text{ for } i = R, 0 \leq k < l < R, \\  
- \sqrt{\frac{1}{N} \mrDifK{k} \sum_{l \geq R} \mrDifK{l}} 
	& \text{ for } i = l = R, 0 \leq k < R. \\	
\end{cases}$$
Now, for $i \leq j < R$, we have
\begin{align*}
\sum_{k=0}^R & \sum_{l>k}^R V_{k,l}^i V_{k,l}^j
= \sum_{k=0}^{R-1} \sum_{l>k}^{R-1} V_{k,l}^i V_{k,l}^j + \sum_{k=0}^{R-1} V_{k,R}^i V_{k,R}^j \\
&= \delta_{ij} \left( \frac{1}{N} \mrDifK{i} \sum_{k=0}^{R-1} (1-\delta_{ik}) \mrDifK{k} \right)
 + (1-\delta_{ij})(- \frac{1}{N} \mrDifK{i} \mrDifK{j})
 + \delta_{ij} \left( \frac{1}{N} \mrDifK{i} \sum_{k \geq R} \mrDifK{k} \right) \\
&= \frac{1}{N} \mrDifK{i} \left( \delta_{ij} - \mrDifK{j} \right)
\end{align*} 
and therefore
\begin{align*}
\sum_{k=0}^R \sum_{l>k}^R V_{k,l}^i V_{k,l}^j \cdot \nu
= [M_i,M_j].
\end{align*}
For $i < j = R$ we get
\begin{align*}
\sum_{k=0}^R \sum_{l>k}^R V_{k,l}^i V_{k,l}^R \cdot \nu
&= - \sum_{k=0}^{R-1} \delta_{ik} \frac{1}{N} \mrDifK{k} \sum_{l \geq k} \mrDifK{l}
 \cdot \nu\\ 
&= - \frac{1}{N} \mrDifK{i} \left(1-\sum_{l=0}^{R-1}\mrDifK{l}\right) \cdot \nu \\
&= - \sum_{l=0}^{R-1}[M_i,M_l] = [M_i,1-\sum_{l=0}^{R-1} M_l] = [M_i,M_R]
\end{align*}
as $[M_i,1] = 0$. Finally, for $i = j = R$,
\begin{align*}
\sum_{k=0}^R \sum_{l>k}^R V_{k,l}^R V_{k,l}^R \cdot \nu
&= \sum_{k=0}^{R-1} \frac{1}{N} \mrDifK{k} \left( 1 - \sum_{l=0}^{R-1} \mrDifK{l} \right) \cdot\nu\\ 
&= \sum_{k=0}^{R-1} \sum_{l=0}^{R-1} \frac{1}{N} \mrDifK{k} \left( \delta_{kl} - \mrDifK{l} \right)
	\cdot \nu \\   
&= \sum_{k,l=0}^{R-1} [M_i,M_l] = [1-\sum_{k=0}^{R-1} M_k,1-\sum_{l=0}^{R-1} M_l] = [M_R,M_R].
\end{align*}
Hence, we can apply the representation theorem, and get independent standard Brownian motions
$(W_{kl})_{0 \leq k < l \leq R}$ such that 
\[ M_i = \sum_{k=0}^{R} \sum_{l>k}^{R} V_{k,l}^i \cdot W_{k,l} \]
for $i \leq R$. Now, assume we have constructed two arrays of standard Brownian
motions $(W_{kl})_{0 \leq k < l \leq R}$ and $(\widetilde{W}_{kl})_{0 \leq k < l \leq \widetilde{R}}$ by the above
construction for $R$ and $\widetilde{R}$ with $R < \widetilde{R}$. As the Brownian motions are
independent within an array, we get
$$ [W_{k,l},W_{m,n}]_t = \delta_{km} \delta_{ln} t = [\widetilde{W}_{k,l},\widetilde{W}_{m,n}]_t $$
for $k,l,m,n < R$. As a consequence of Lévy's characterization of the Brownian motion, 
$(W_{kl})_{0 \leq k < l < R}$ and $(\widetilde{W}_{kl})_{0 \leq k < l < R}$ are equal in
distribution. Hence, their projective limit for $R \to \infty$ exists in form of independent
standard Brownian motions $(W_{kl})_{0 \leq k < l < \infty}$ with
$$ M_i = \sum_{k=0}^{\infty} \sum_{l>k}^{\infty} V_{k,l}^i \cdot W_{k,l}. $$
Substituting this into \eqref{da:eq:defMi} finishes the proof.
\end{proof}


\subsection{Existence And Uniqueness Of A Solution}

Now with Theorem~\ref{da:t:SDE=MP} at hand, we can prove that SDE~\eqref{da:eq:difdef} is well-posed
by verifying the same for the martingale problem. Here, a corresponding result
is known for bounded generators (\cite{hennequin_measure-valued_1993}). However, for Muller's ratchet $\GAsel$ is not
bounded for $\alpha > 0$. We therefore use a Girsanov theorem to reduce the general system to this
case. First, we state the well-posedness of the SDE as the main result we want to prove in this
section.

\begin{Theorem}[Well-posedness of the SDE]
\label{da:t:eau}
Let  $\mathbf{P}_0 \in \mathcal{P}(\SimplexEM)$ and $\mrDifT{0} \sim \mathbf{P}_0$ with
$\mathbf{E}_{\mathbf{P}_0}\left[ h_\xi(\mrDifT{0}) \right] < \infty$ for all $\xi >0$. 
Then, for $N\in (0,\infty)$ and $\alpha,\lambda,\gamma\in [0,\infty)$, the
system~\eqref{da:eq:difdef} starting in $\mrDifT{0}$ has a unique weak solution $\mrDif = (\mrDifT{t})_{t\geq 0}$ in the space
$\mathcal C_{\SimplexEM}([0,\infty))$ of continuous functions on $\SimplexEM$. We refer to the
process $\mrDif$ as \emph{diffusion approximation of Muller's ratchet} (with
compensatory mutations) with selection coefficient $\alpha$, mutation rate
$\lambda$, compensatory mutation rate $\gamma$ and population size $N$.
\end{Theorem}

\noindent 
In terms of martingale problems, this theorem translates to the following proposition.

\begin{Proposition}[Well-posedness of the martingale problem]
\label{da:p:mp-wp}
Let $\mrDifT{0}$ and $\mathbf{P}_0$ be as in Theorem~\ref{da:t:eau}, $\GA$ as in
Definition~\ref{da:d:D&G-MP}, $\alpha, \lambda, \gamma\in [0,\infty)$, $N\in (0,\infty)$ and
$\overline{\mathcal F}$ be as in Definition~\ref{pre:d:F}. Then, the $(\mathbb S^\circ,\mathbf P_0,
G_{\mathcal X}^\alpha, \overline{\mathcal F})$-martingale problem is well-posed and has continuous
paths on $\SimplexEM$. Its solution is equal in distribution to the diffusion approximation of
Muller's ratchet with compensatory mutations.
\end{Proposition}

\noindent
For the proof of Proposition~\ref{da:p:mp-wp}, we need two well-known results from the theory of
semimartingales. Briefly recall that a sequence of stopping times $(\tau_n)_{n \in \N}$ is said
to be a \emph{localizing sequence} if $\tau_n \nearrow \infty$ almost
surely. An adapted process $M$ is said to be a \emph{local martingale} if there
exists a localizing sequence such that the stopped process $M^{\tau_n}$ is a
martingale for every $n$. Obviously, every martingale is a local martingale,
while the reversal is not true in general, but under the following condition.

\begin{Lemma} \label{da:l:lm}
Let $\mathcal N = (N_t)_{t\geq 0}$ be a local martingale. If
$\mathbf E[\sup_{0\leq t\leq T}N_t]<\infty$ for all $T>0$, then
$\mathcal N$ is a martingale.
\end{Lemma}
\begin{proof} 
See e.g.\ \citet{protter_stochastic_2005}, Theorem~I.51.
\end{proof}

\noindent
The second result is a Girsanov type theorem. It is the key step in the proof of
Proposition~\ref{da:p:mp-wp}, as it will allow us the reduce the martingale problem to the case
without selection, where the generator is bounded.

\begin{Theorem}[Girsanov Theorem for continuous semimartingales] \label{da:t:grisanov}
If $\mathcal L=(L_t)_{t\geq 0}$ is a continuous $\mathbf P$-martingale for some probability measure
$\mathbf P$, then $\mathcal Z = (Z_t)_{t\geq 0}$, given by 
$Z_t = e^{L_t - \tfrac 12 [\mathcal{L}]_t}$, is a continuous local martingale. 

\noindent
If $\mathcal Z$ is a martingale as well, $\mathcal N = (N_t)_{t\geq 0}$ is a $\mathbf P$-local
martingale and $\mathbf Q$ is defined via
$$ \frac{d\mathbf Q}{d\mathbf P}\Big|_{\mathcal F_t} = Z_t,$$
then $\mathcal N - [\mathcal L, \mathcal N]$ is a $\mathbf Q$-local martingale. 
\end{Theorem}
\begin{proof} 
See e.g. \citet{kallenberg_foundations_1997}, Theorem~16.19 and Lemma~16.21.
\end{proof}

\noindent
Combining Lemma~\ref{da:l:lm} and the Girsanov theorem, we can show now that a
solution of the martingale problem for selection coefficient $\alpha$ is also a
solution for the one with selection coefficient $\alpha' > \alpha$ on a modified
probability space.

\begin{Proposition}[Change of measure] \label{da:p:com}
For  $\Vector{x} \in \Simplex$, we denote with
\begin{align}\label{da:eq:kappa12}
\kappa_1(\Vector{x}) := \sum_{k=0}^\infty k x_k \quad \text{and} \quad
\kappa_2(\Vector{x}) := \sum_{k=0}^\infty (k - \kappa_1(\underline x))^2 x_k
\end{align}
the expectation and variance of $\Vector{x}$. 
Now let $\Vector{x} \in \SimplexEM$ and $\mrDif = (\mrDifT{t})_{t\geq 0}$ be a solution of the
$(\SimplexEM, \delta_{\underline x}, \GA, \overline{\mathcal{F}})$-martingale problem and denote its
distribution by $\mathbf{P}^\alpha$. Then, the process $\mathcal Z^{\alpha,\alpha'} =
(Z^{\alpha,\alpha'}_t)_{t\geq 0}$, given by
\begin{equation}
\begin{aligned}\label{da:eq:Zt}
Z^{\alpha,\alpha'}_t = \exp \! \Big( N(\alpha - \alpha')
\Big(&\kappa_1(\underline X(t)) - \kappa_1(\underline X(0)) \\
& %\qquad \qquad \qquad \qquad \qquad \qquad
- \int_0^t \lambda - \gamma \kappa_1(\underline X(s)) - \frac{\alpha + \alpha'}{2}
\kappa_2(\underline X(s)) \, ds \Big)\!\Big)
\end{aligned}
\end{equation}
is a $\mathbf P^\alpha$-local martingale. If $\alpha'>\alpha$, it is even a $\mathbf
P^\alpha$-martingale and the probability measure $\mathbf P^{\alpha'}$, defined by
$$ \frac{d \mathbf P^{\alpha'}}{d \mathbf P^\alpha}\Big|_{\mathcal F_t} = Z^{\alpha,\alpha'}_t,$$
solves the $(\SimplexEM, \delta_{\underline x}, G^{\alpha'}, \overline{\mathcal F}))$-martingale problem.
\end{Proposition}

\begin{Remark}
The quantities $\kappa_1(\Vector{x})$ and $\kappa_2(\Vector{x})$ are called the
first and second cumulant of $\Vector{x}$. Cumulants have proven to be useful
for Muller's ratchet. We will take a closer look at them in Section~\ref{ode:sec:cumulants}.
\end{Remark}

\noindent
\begin{proof}[Proof of Proposition~\ref{da:p:com}]
We know from Theorem~\ref{da:t:SDE=MP} that $\mrDif$ satisfies the assumptions of
Lemma~\ref{da:l:expBounds}.  The proof is an application of the Theorem~\eqref{da:t:grisanov}. 
By assumption and Lemma~\ref{pre:l:conti} and Lemma~\ref{pre:l:conti_paths}, the processes $\mrDif$ and
$f(\mrDif)$ are continuous for all $f\in\overline{\mathcal F}$.
Let again $\mathcal N^f = (N^f_t)_{t\geq0}$ be
$$ N^f_t := f(\underline X(t)) - f(\underline X(0)) - \int_0^t G_{\mathcal X}^\alpha f(\underline
X(s))ds.$$ Now for
$$ g(\Vector{x}):= N(\alpha - \alpha')\kappa_1(\Vector{x}) \in \overline{\mathcal{F}} $$ 
we know that $\mathcal L = (L_t)_{t\geq 0}$, defined by
\begin{align*}
L_t &:= N^g_t 
= N(\alpha - \alpha') \Big(\kappa_1(\mrDifT{t}) - \kappa_1(\mrDifT{0}) - \int_0^t \GA
    \kappa_1(\mrDifT{s}) ds \Big)  
%&= 2N(\alpha - \alpha') \Big(\kappa_1(\mrDifT{t}) - \kappa_1(\mrDifT{0})-
%    \int_0^t \lambda - \alpha \kappa_2(\mrDifT{s}) - \gamma \kappa_1(\mrDifT{s}) ds\Big).
\end{align*}
is a $\mathbf{P}^\alpha$-(local) martingale. As
\begin{align*}
\GAselFX{\kappa_1}{\Vector{x}}
&= \alpha \sum_{k=0}^\infty (\kappa_1(\Vector{x}) - k) x_k k 
= \alpha \left( \kappa_1^2(\Vector{x}) - \sum_{k=0}^\infty k^2 x_k \right) 
= -\alpha \kappa_2(\Vector{x}), \\
\GAmutFX{\kappa_1}{\Vector{x}}
&= \lambda \sum_{k=0}^\infty ( x_{k-1} - x_k ) k 
= \lambda \sum_{k=0}^\infty (k+1) x_k - k x_k
= \lambda, \\
\GAcompFX{\kappa_1}{\Vector{x}}
&= \gamma \sum_{k=0}^\infty \left( (k+1)x_{k+1} - k x_k \right) k
= \gamma \sum_{k=0}^\infty (k-1) k x_k - k^2 x_k
= - \gamma \kappa_1(\Vector{x}),
\end{align*}
$\GNresFX{\kappa_1}{\Vector{x}} = 0$ and 
\begin{align*}
\GNresFX{\kappa_1^2}{\Vector{x}} 
= \frac{1}{2N} \sum_{l,k=0}^\infty x_k (\delta_{kl}-x_l) 2 k l
= \frac{1}{N} \sum_{k=0}^\infty x_k (k^2-k \kappa_1(\Vector{x}))
= \frac{1}{N} \kappa_2(\Vector{x}),
\end{align*}
we have
\begin{align*}
[\mathcal{L}]_t 
= N^2(\alpha-\alpha')^2 \int_0^t \GNresFX{\kappa_1^2}{\mrDifT{s}} ds 
= N(\alpha-\alpha')^2 \int_0^t \kappa_2(\mrDifT{s}) ds
\end{align*}
and
\begin{align*}
&\exp\! \left( L_t - \frac{1}{2}[\mathcal{L}]_t \right) \\
&\, = \exp \! \Big( N (\alpha - \alpha') \Big(\kappa_1(\mrDifT{t}) - \kappa_1(\mrDifT{0})
		 		- \int_0^t \lambda - \gamma \kappa_1(\mrDifT{s}) - \frac{\alpha + \alpha'}{2}
		 		\kappa_2(\mrDifT{s}) ds \Big)\!\Big) \\
&\, = Z^{\alpha,\alpha'}_t
\end{align*}
is a local $\mathbf{P}^\alpha$-martingale by Theorem~\ref{da:t:grisanov}. Now, if 
$\alpha<\alpha'$ and $\mrDifT{0}$ has all exponential moments, we have that $\mathbf
E[\sup_{0\leq t\leq T} Z^{\alpha,\alpha'}_t]<\infty$ since $e^{\xi \kappa_1(\underline X(t))} \leq
h_\xi(\underline X(t))$ by Jensen's inequality for all $\xi$. Hence, $\mathcal Z^{\alpha, \alpha'}$
even is a $\mathbf P^\alpha$-martingale by Lemma~\ref{da:l:lm}.

\noindent
Now let $f \in \mathcal{\overline F}$. By assumption, $N^f$ is a $\mathbf{P}^\alpha$-martingale. 
We calculate
\begin{align*}
[\mathcal L, \mathcal N^f]_t 
& = \int_0^t \GNresFX{gf}{\mrDifT{s}} - g(\mrDifT{s}) \GNresFX{f}{\mrDifT{s}} ds \\ 
& = \frac{(\alpha-\alpha')}{2} \int_0^t \sum_{k,l=0}^\infty X_k(s)(\delta_{kl} -
    X_l(s)) 
    \left( l \frac{\partial}{ \partial x_k} f + k \frac{\partial}{ \partial x_l} f \right)
    ds \\ 
& = \frac{(\alpha-\alpha')}{2} \int_0^t 
 	\Big( 
 		\sum_{k=0}^\infty \mrDifKT{k}{s} \big(k - \kappa_1(\mrDifT{s})\big)
     	\frac{\partial}{\partial x_k} f(\mrDifT{s}) \\
&\qquad \qquad \qquad \qquad \qquad \qquad + 
     	\sum_{l=0}^\infty \mrDifKT{l}{s} \big(l - \kappa_1(\mrDifT{s})\big)
     	\frac{\partial}{\partial x_l} f(\mrDifT{s}) \Big) 
     ds \\
& = \int_0^t \Gsel{\alpha'}f(\mrDifT{s}) - \Gsel{\alpha}f(\mrDifT{s}) ds.
\end{align*}

\noindent
Therefore the Girsanov theorem for continuous semimartingales shows that
\begin{align*}
N_t^f - \left[ \mathcal{L}, \mathcal{N}^f \right]_t 
= f(\mrDifT{t}) - \int_0^t G^{\alpha'} f(\mrDifT{s}) ds
\end{align*}
is a $\mathbf P^{\alpha'}$-martingale. Hence, $\mathbf P^{\alpha'}$ solves
the $(\mathbb S^\circ, \mathbf P_0, G_{\mathcal X}^{\alpha'},\overline{\mathcal F})$-martingale problem.
\end{proof}

\noindent
Using this result, we can now prove the well-posedness of the stochastic
differential equation \eqref{da:eq:difdef}.

\begin{proof}[Proof of Theorem~\ref{da:t:eau}]
The assertions of Theorem~\ref{da:t:eau} and Pro\-position~\ref{da:p:mp-wp} are
equivalent, as can be seen from Theorem~\ref{da:t:SDE=MP}. So, it
suffices to show Proposition~\ref{da:p:mp-wp}. Moreover, it is enough to
consider the case $\mathbf P_0 = \delta_{\underline x}$ for
$\underline x\in \SimplexEM$.

For $\alpha=0$, it can be seen from classical theory
(e.g.~\citealp[Theorem 5.4.1]{hennequin_measure-valued_1993}) that the 
$(\mathbb S,\delta_{\underline x}, G^0, {\mathcal F})$-martingale
problem has a unique weak solution. Thus, by Theorem~\ref{da:t:SDE=MP}, the 
$(\mathbb S,\delta_{\underline x}, G^0, \overline {\mathcal F})$-martingale 
problem is well-posed as well. Denote the unique distribution which solves 
the martingale problem by $\mathbf P^0$. Hence, by Proposition~\ref{da:p:com}, 
there is a change of measure using $\mathcal Z^{0,\alpha}$ from $\mathbf P^0$ 
to $\mathbf P^\alpha$ and $\mathbf P^\alpha$ solves the $(\mathbb S^\circ, \mathbf P_0, \GA,
\overline{\mathcal F})$-martingale problem. This establishes the existence.

For uniqueness, recall that the solution of the 
$(\mathbb S,\delta_{\underline x}, G^0, {\mathcal F})$-martingale
problem is unique. We proceed by contradiction and assume that
$\mathbf P_1^\alpha$ and $\mathbf P_2^\alpha$ are two different
solutions of the $(\SimplexEM, \delta_{\underline x},\GA, \overline{\mathcal F})$-martingale
problem. Let $\tau_1, \tau_2,...$ be a localizing sequence such that
$(Z^{\alpha,0}_{t\wedge \tau_n})_{t\geq 0}$, given by~\eqref{da:eq:Zt}, is both a 
$\mathbf P_1^\alpha$-martingale and a $\mathbf P_2^\alpha$-martingale. Since
$\mathbf P_1^\alpha \neq \mathbf P_2^\alpha$, there must be $t\geq 0$ such that the distribution of
$\underline X(t)$ under $\mathbf P_1^\alpha$ and $\mathbf P_2^\alpha$ is different (see Theorem 4.4.2
in \cite{ethier_markov_2005}). Hence, there is $f\in{\mathcal F}$
and $n\in\N$ such that $\mathbf E_{\mathbf P_1^\alpha}[f(\mrDifT{t\wedge \tau_n})] \neq \mathbf
E_{\mathbf P_2^\alpha}[f(\mrDifT{t\wedge \tau_n})]$ as well as
  \begin{align}
    \label{da:eq:unequal}
    \mathbf E_{\mathbf P_1^\alpha}[Z_{t\wedge
      \tau_n}^{\alpha,0}f(\mrDifT{t\wedge \tau_n})] \neq \mathbf
    E_{\mathbf P_2^\alpha}[Z_{t\wedge \tau_n}^{\alpha,0}f(\mrDifT{t\wedge \tau_n})].
  \end{align}
  However, by the same arguments as in the proof of
  Proposition~\ref{da:p:com}, $Z_{t\wedge \tau_n}^{\alpha,0} \cdot \mathbf
  P_1^\alpha$ as well as $Z_{t\wedge \tau_n}^{\alpha,0} \cdot \mathbf
  P_2^\alpha$ equal $\mathbf P^0$ on the $\sigma$-algebra
  $\sigma((\mrDifT{s})_{0\leq s\leq t\wedge \tau_n})$. Hence,
  $$\mathbf E_{\mathbf P_1^\alpha}[Z_{t\wedge
    \tau_n}^{\alpha,0}f(\mrDifT{t\wedge \tau_n})] = \mathbf
  E_{\mathbf P^0}[f(\mrDifT{t\wedge \tau_n})] = \mathbf
  E_{\mathbf P_2^\alpha}[Z_{t\wedge \tau_n}^{\alpha,0}f(\mrDifT{t\wedge \tau_n})],$$ in contradiction
  to~\eqref{da:eq:unequal}. Uniqueness of the $(\mathbb S^\circ, \mathbf
  P_0, G_{\mathcal X}, \overline{\mathcal F})$-martingale problem
  follows.
\end{proof}