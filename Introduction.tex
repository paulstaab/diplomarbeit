\chapter{Introduction}

\begin{quote}
``I will not attempt here to predict from theory the quantitative effect of the
ratchet mechanism. Involving natural selection, mutation, and genetic drift at many linked loci, the problem poses
enormous difficulties for the application of population genetics theory. But the possible
significance of the phenomenon makes it important that some theoretical treatment should be
attempted. The ratchet mechanism has been unjustly ignored by theoretical population genetics.''
\cite{felsenstein_evolutionary_1974}  
\end{quote}

\noindent
While the problems Joseph Felsenstein saw arising from a mathematical model in population
genetics back in 1974 have not been solved to the present day, \emph{Muller's ratchet} has 
in the meanwhile gotten the attention of biomathematicians and has become a
well-known model in population genetics.

The importance he claimed it has is given by the biological question to which
Muller's ratchet is a possible answer:
\begin{quote}Why has sex evolved?\end{quote}
From an evolutionary point of view, the step from asexual to sexual reproduction
is a rather harsh one, as the population suddenly needs to spend a significant
part of its resources on individuals that only contribute a minor part to
reproduction: males. Population geneticists have therefore searched for an
advantage that justifies this effort and have found a biological phenomenon
called (sexual) recombination: During meiosis, a process that is part of the production
of gametes, either the maternal or the paternal version of every chromosome is
transferred to the gamete. As this happens independently for each chromosome,
(nearly) all gametes contain a mixture of the genomes of their producer's 
parents.\footnote{To be precise, recombination refers to the ``shuffling''
of genetic material in general. The described phenomenon is just one form of
recombination, the so called interchromosomal recombination. While there are
other important mechanisms leading to recombinations (e.g. crossing-overs), it
suffices for our purpose to imagine it in the described way.}

It has already been suggested by \cite{fisher_genetical_1930} and
\cite{muller_genetic_1932} that the possibility of recombinations may be an
advantage of sexual reproduction because beneficial mutations -- small
spontaneous changes in the genome that increase the ``fitness'' of an individual
-- can be combined in a later descendant, even if they occur in different individuals. 
While the importance of this effect was
discussed heavily\footnote{See \cite{felsenstein_evolutionary_1974} for a
summary.}, \cite{muller_relation_1964} gave another argument looking at
mutations that decrease an individuals fitness: Assume that such deleterious
mutations constantly occur inside a population. As genomes are typically large, there is
only a very small chance that two different mutations will occur in the very
same place in the genome. Hence, the reversion of one mutation by another one is so
unlikely that we can ignore this effect. Without recombination, an
individual passes its whole genome -- including all mutations -- to all its
offspring. Thus, the total number of mutations in the population can decrease
only if individuals with many mutations have no descendants. However, once all
mutants have by chance acquired one mutation, there will never be an individual
free of mutations again. As this will happen again and again, the population
will gather more and more deleterious mutations and will finally face
extinction. Because the accumulation of mutations happens in an 
irreversible way, like the ``clicks" of a ratchet moving forward notch by
notch, this argument became known as Muller's ratchet.

This argument was qualitatively accepted in the following years
and has by now become a textbook model for the advantage of sexual reproduction
in the biological literature. Even though Joseph Felsenstein proved to be
right in the difficulties he had seen in determining the quantitative effect of
Muller's ratchet. Up to now the literature concerned with Muller's ratchet
relies on computer simulations and approximating models to determine how fast
the fitness of a population decreases (e.g. \cite{haigh_accumulation_1978},
\cite{lynch_mutation_1990}, \cite{stephan_advance_1993},
\cite{gessler_constraints_1995}, \cite{gordo_degeneration_2000},
\cite{etheridge_how_2008}). Though this is the key question in the analysis of
Muller's Ratchet, an exact determination of this ``click rate'' still seems to
be out of reach.

Along with other extensions of the classical model, like a diploid version of
Muller's ratchet (\cite{pamilo_accumulation_1987},
\cite{charlesworth_rapid_1997}) and a model with variable population size
(\cite{gabriel_mullers_1993}), the behavior of Muller's ratchet with so called
\emph{compensatory mutations} has been discussed in the past. In contrast to the
normal model, we relax the above mentioned assumption that deleterious
mutations cannot be compensated by a second mutation. Hence, there is a
small chance for the occurrence of a back-mutation. We assume that the
probability for such a compensatory mutation is $\gamma$ for every malicious
mutation in the genome. However, deleterious mutations typically arise with a
rate of $\lambda$ per genome, whereas the rate of compensatory mutations only
scales with the number of acquired mutations. \cite{haigh_accumulation_1978}
concluded that compensatory mutations occur far too rarely to counter the
ratchet effect. However, as the rate of compensatory mutations increases
with the number of acquired deleterious mutations -- while the rate of occurrence of the
latter is constant --  it seems reasonable that the ratchet will eventually
reach an equilibrium state if compensatory mutations are present.

It has been discussed if this equilibrium can be reached by natural populations,
or if it lies above the maximal number of deleterious mutations that a
population can bear. On the one hand, \cite{antezana_era_1997} have argued that
the former could be important for small viruses. On the other hand,
\cite{loewe_quantifying_2006} concluded that the latter is likely for human
mitochondria and \cite{chao_fitness_1990} and \cite{smith_evolution_1978}
claimed that compensatory mutations occur far too rarely to have a significant
effect in a population undergoing Muller's ratchet. However, new insights in
molecular biology in the past decades have shown that (back-)mutation is not the
only mechanism in living cells that can compensate the effects of unfavorable
mutations. \cite{maisnier-patin_adaptation_2004} suggested at least five
different ways how this could theoretically happen. \cite{poon_rate_2005} found
experimental evidence of such compensations in the DNA bacteriophage
$\varphi$X$174$. \cite{maier_complex_2008} and
\cite{depperschmidt_mathematical_2011} studied the plasmid genomes of mosses and
concluded that deleterious mutations are compensated by RNA editing, a couple of
mechanisms by which specific bases in the genome are altered after the sequence
has been transcribed from DNA to RNA.

Summarizing all these mechanisms, it is reasonable to assume that the rate of
compensatory mutations can be high enough to affect a natural population.
Therefore we extend the classical model of Muller's ratchet by compensatory
mutations and extend most of the known results about the ratchet to that case.
In particular, we show that the ratchet with compensatory mutations has exactly one
equilibrium state in which the malicious effects of mutations on one side
cancels out with the beneficial effects from purifying selection and
compensatory mutations on the other (Corollaries~\ref{ode:c:ex_stat_point} \&
\ref{ode:c:conv_to_theta}). Furthermore, the population always reaches
this equilibrium as the time goes towards infinity (also Corollary
\ref{ode:c:conv_to_theta}). To achieve these results, we start with some
necessary topological preliminaries (Chapter~$2$). Thereafter, we state the
corresponding results for the classical ratchet, where they can be obtained in a
time-discrete model, mostly due to the work of \cite{maia_analytical_2003}
(Chapter~$3$). In Chapter $4$, we introduce compensatory mutations and start
with a brief look at an extended version of Haigh's time-discrete model
(Chapter~$4.1$). Afterwards we move on to time-continuous models. Here we
will use the diffusion approximation of Muller's ratchet, which is based on a
stochastic differential equation (SDE). We translate the SDE into a martingale
problem and show that it uniquely defines a stochastic process (Chapter~$4.2$).
Afterwards we increase the population size towards infinity, were the SDE turns
into an ordinary differential equation (ODE). We solve the ODE and derive the
mentioned results (Chapter~$4.3$). Finally, we present computer simulations
to see how large the ``finite population effect'' is for various combinations
of parameters (Chapter~$5$).